% Created 2017-09-17 Sun 01:08
% Intended LaTeX compiler: pdflatex
\documentclass[a4paper,11pt,titlepage]{article}
\usepackage[utf8]{inputenc}
\usepackage[T1]{fontenc}
\usepackage{graphicx}
\usepackage{grffile}
\usepackage{longtable}
\usepackage{wrapfig}
\usepackage{rotating}
\usepackage[normalem]{ulem}
\usepackage{amsmath}
\usepackage{textcomp}
\usepackage{amssymb}
\usepackage{capt-of}
\usepackage{hyperref}
\author{Prasanna Rajaram}
\date{\today}
\title{Power optimization by load monitoring and control}
\hypersetup{
 pdfauthor={Prasanna Rajaram},
 pdftitle={Power optimization by load monitoring and control},
 pdfkeywords={},
 pdfsubject={},
 pdfcreator={Emacs 26.0.50 (Org mode 9.1)}, 
 pdflang={English}}
\begin{document}

\maketitle
\tableofcontents

\section{Introduction}
\label{sec:orgfb07e53}
In conventional power saving techniques, the user has to be vigilant
in turning ON/OFF electrical loads as and when required. This ensure
optimal consumption and reduce power bill for the consumer. This
effort can be automated and optimized better. This project will enable
the user to control loads from anywhere as long as the mobile device
and the Raspberry Pi (@ home) are connected to Internet.  

The intention is to retrofit this system into the existing domestic
wiring scheme of the user with minimal changes as much as possible.

\subsection{Purpose of this document}
\label{sec:org22e2998}
The purpose of this document is to list down the design specification
and design details for this home automation project. 

\subsection{Objectives}
\label{sec:org2c52611}
\begin{enumerate}
\item The project will enable users to switch electrical loads through a
mobile application
\item The load ON/OFF times shall be logged. This logged information can
be used to provide the user with suggestions for power saving.
\item Using interfaces like gas detectors, one can log information if
there were any gas leakage in any one of the apartments using
centralized logging system.
\item Power factor correction for inductive loads and switching in/out
"tank circuits" shall also be done based on the load conditions
\item 

\item \ldots{}\ldots{}.
\end{enumerate}

\section{Design}
\label{sec:org95caea1}

\section{Wiring scheme}
\label{sec:org413352c}
In the conventional power switching arrangement, the following
modifications should be made to implement this design. The SPST switch
used conventionally to switch ON/OFF loads, should be replaced with a
DPST switch (commonly known as "two-way" switch). A relay with DPST
switching arrangement should be connected to the new DPST switch. This
two-way switch wiring scheme established by the new two-way switch and
the DPST portion of the relay allows the user to switch ON/OFF loads
either from the two switch or by microcontroller. The relay shall be
driven by a Arduino microcontrolller. The wire leading to the load
from the relay contact shall have an ACS712 (Hall effect sensor)
connected in series before the load. The pupose of the Hall effect
sensor is to measure the current drawn by the load. The output of the
Hall effect sensor shall be wired as input to one of the analog
channels of the Arduino controller.  The wiring drawing is presented below

\#\#\#\#\# Wiring diagram - insert here \#\#\#\#\#

\section{High level block diagram}
\label{sec:orge222f56}
The following portions shall make up this design
\begin{enumerate}
\item Wiring scheme with modified switch, relay and Hall effect sensor
\item Arduino (Micro) micro controller
\item Raspberry Pi
\end{enumerate}

\subsection{Arduino micro controller}
\label{sec:org469cd2b}
The microntroller will be able to see the ON times of all the loads if
either they are switched via the controller itself or directly by the
DPST switch. This information is logged at the microntroller level and
passed on to Raspberry Pi. 
\subsection{Raspberry Pi}
\label{sec:orgb4bce5f}
\subsubsection{HMI}
\label{sec:org79aa924}
The Pi is responsible for pulling all the data logged by the
microcontroller and present it on screen for the user to review. The
Pi functions like an HMI. A small 7" touch enabled screen could be
wired to RPi to enable users to control loads and view reports locally
\subsubsection{Data analysis}
\label{sec:org99d6979}
The next function of Pi is to do statistical analysis of the logged
information. This analysis will come out with results that could
provide suggestions to the user with load profiles. This report will
tell the user how much power was consumed by respective loads. It will
also present the normal ON times of the load during a day. Based on
this data, the user can take an informed decision about turning OFF
loads at certain time of the day.

\subsection{RPi Alternative}
\label{sec:org9d4aded}
Provided the Arduino is able to communicate directly to a server over
the internet (using Arduino Wifi Shield), then all data analysis and
HMI functions can be passed on to application in the server
machine. Data can be pulled over the network to a mobile device to
control and perform HMI functions.
\subsection{HMI UI Design}
\label{sec:org03dafd8}
It is envisaged to have a one screen display which will display all
types of loads on the first screen. The groups could be like lights,
fans, air conditioners, heaters etc.  When the user taps on one of the
groups (say lights), the screen would expand to a list or table
(list/tabulate all lights with their statuses). From this expanded
page, the user would be able to toggle the load status.  

This can further be improved based the users' usage pattern. For
example, in day time of the hot months, the user would want to keep
the fans running and my turn ON/OFF air conditioner based on
requirement. In this case, (assuming) good ambient lighting from the
sun, the user may never turn ON the lights during the day in the
balcony / kitchen / hall. So depending the season of the year, time of
day the list could dynamically display only the frequently used loads
for control besides the exhaustive list.

\section{Some special controls}
\label{sec:org3d50bea}
Assuming that a humidity sensor and temperature sensor module are
wired into the Arduino, we can collect humidity, temperature and air
conditioner "turn-ON" times information based on user behaviour. This
information could be used to predict when the user would want to have
the air conditioner turned-ON/OFF from the humidity and temperature
sensor readings.

\section{Bill of material}
\label{sec:orgbc063b6}
\begin{center}
\begin{tabular}{rclll}
S.No & Mandatory & Part No. & Model Name & Approx. Cost\\
 & /Optional &  &  & \\
\hline
1 & M &  & Arduino Micro & \\
2 & M &  & Arduino Wifi Shield & \\
3 & O &  & Raspberry Pi & \\
4 & M &  & Relay board (X relays) & \\
5 & M &  & Hall effect sensor & \\
6 & O &  & Humidity sensor & \\
7 & O &  & Temperature sensor & \\
8 & O &  & Gas (LPG) detector & \\
9 & M &  & Arduino Wifi Shield & \\
10 & M &  & Breadboard & \\
11 & M &  & Power supply +5V 2A & \\
12 & M &  & Resistors & \\
 &  &  & - 220 Ω & \\
 &  &  & - 10k Ω & \\
 &  &  & - \ldots{} Ω & \\
13 & M &  & Capacitors & \\
 &  &  & - 10 µF & \\
\end{tabular}
\end{center}
\end{document}
